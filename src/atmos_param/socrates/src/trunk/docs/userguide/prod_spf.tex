
Before the main code can be run it is necessary to build a spectral file.
A number of programs are used in building the spectral file and detailed
descriptions are given in the next chapter. These programs can be run 
interactively or using UNIX scripts. This process is somewhat involved:
it may be most helpful to read the general discussion here first, then to
read carefully through the examples in the directory {\tt \$RAD\_DIR/examples}
making continual reference to the descriptions of programs found in the next
chapter.

The first step in this process is to design a spectrum, choosing the number
of spectral bands and the wavelengths which delimit them, the absorbers
which are active in each of the bands, the possible types of continua, and
the types of aerosols which are to be included. These data are entered into a 
skeletal spectral file using the program {\tt prep\_spec} (see the example
in section~\ref{sec:ex_spf}).

Having created the skeletal spectral file, the spectral data for 
different spectral
processes have to be generated. The procedure for each process is
self-contained and there is no need to consider processes in any particular
order, except that gaseous absorption and continuum absorption must
be considered together.

The correlated-$k$ method may be used to
generate data for line and continuum absorption. There is one program, 
{\tt corr\_k} which generates both the line and the continuum data, though
it must be run separately for each process. An example use of this
program can be found in {\tt \$RAD\_DIR/examples/corr\_k}.

Aerosols, cloud droplets and ice crystals all introduce scattering effects
for which 
parametrizations are required. The scattering properties of spherical
particles may be calculated from a Mie scattering code. For non-spherical
ice crystals, the use of predefined databases of propeties is preferred.
The scattering code for spherical particles is called {\tt scatter}. This
is a fairly general code which will generate single scattering properties at
a range of specified wavelngths, integrated over a distribution of sizes.
In the case of aerosols a parametrization of the effects of humidity due
to D.~L.~Roberts is available. The normal scattering algorithm is a Mie
scattering code based on that given by Bohren and Huffman. Data for ice
crystals may be generated from this code if they are modelled as spheres,
but in general databases of scattering propeties should be used.

These monochromatic single-scattering data, whether generated from the Mie
code or from the non-spherical code, must be averaged across bands and
possibly parametrized before being included in the spectral file. The
output from several runs of the scattering program may be concatenated to
make one large file for input to the program for averaging scattering data,
{\tt scatter\_average},
each sub-file being termed a block: this makes it possible, for example, to
feed in data generated with different effective radii for fitting. The 
program reads each block of data, averages the single scattering properties
across the bands of a spectral file and then fits a parametrization to these
averaged properties if required.

Once these files have been generated, the data must be inserted into the
spectral file: this is also done with the program {\tt prep\_spec} which allows
the user to append data to an existing spectral file, or to generate a new
file from an old one. On entry into the program the user is presented with
a menu of block numbers from which to select an action to make blocks of
that number. As well as providing gaseous or scattering data if they are 
required, the user can calculate the fraction of
the solar spectrum in each band, the Rayleigh scattering coefficients, the
surface properties, or a polynomial fit to the Planck function in each
band as required. There is no necessity to produce blocks of every type:
for example, if one did not wish to consider Rayleigh scattering in the
radiation code, there would be no need to calculate the scattering
coefficients. Of course, for solar calculations the fraction of the solar
spectrum in each band must be calculated, and for infra-red calculations a
fit to the Planck function is required. It is perfectly possible to add to
a spectral file at a later date, so one might have a spectral file including
gaseous and droplet data and now wish to add aerosol data: this file could
be used together with Mie data to run {\tt scatter\_average} to generate
averaged aerosol scattering properties, and these properties could then be
appended to the file. The minimum permissible content of a spectral file is
that which is produced when {\tt prep\_spec} is first run. We have previously
called this the skeletal spectral file, though this term should not be taken to
imply that it has any special and peculiar form.

Once the spectral file has been produced the radiation code can be run.
Note again that it is not necessary to generate a spectral file every time
the code is to be run. They may be kept in a library and used as appropriate.
Some standard files are kept in the directory {\tt \$RAD\_DATA/spectra}.
