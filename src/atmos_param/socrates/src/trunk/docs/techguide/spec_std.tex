\section{Standard Spectral Files}

Standard spectral files that have been used operationally for numerical
weater prediction or climate runs are described. Spectral files with names
beginning spec\_ or spec3a\_ are namelist files available for UM versions
up to 8.5 (readable text equivalents of these files are available with the
offline Edwards-Slingo code). Files with names beginning sp\_ are readable
text versions for use with UM versions 8.6 onwards or the offline code.

\subsection{Global Atmosphere Configuration 7}

\subsection*{Spectral file: sp\_sw\_ga7}

Sections are identical to sp\_sw\_ga3\_0 except for changes to the spectral bands, solar spectrum (including Rayleigh coefficients), and gaseous absorption:


\subsubsection*{Spectral bands}

The six spectral bands are identical to sp\_sw\_ga3\_0 except the combined bands 2 and 3 are now properly split into two true bands at 505nm. Band limits are now:

\begin{tabular}{l|l}
Band & Wavelength (nm)\\ \hline
1 & 200 - 320\\
2 & 320 - 505\\
3 & 505 - 690\\
4 & 690 - 1190\\
5 & 1190 - 2380\\
6 & 2380 - 10000\\
\end{tabular}


\subsubsection*{Solar spectrum}

New solar spectrum ("lean\_12") taken as a mean of the spectral data from 2000-2011 from the recommendation of the SPARC/SOLARIS group (data from Judith Lean, available here: http://solarisheppa.geomar.de/ccmi). Associated updates to Rayleigh scattering coefficients. 

\subsubsection*{Gaseous absorption}

Newly derived gaseous absorption for all gases based on HITRAN 2012 and CAVIAR water vapour continuum. Scaling of absorption coefficients uses a look-up table of 59 pressures with 5 temperatures per pressure level based around a mid-latitude summer profile.

Addition of N2O and CH4 minor gases.

Ozone cross sections for the UV and visible come from \citet{Serdyuchenko2014} and \citet{Gorshelev2014} (with Brion-Daumont-Malicet cross-sections for the far UV) taken from this website: http://igaco-o3.fmi.fi/ACSO/cross\_sections.html. In band 1, a single k-term is calculated for each 20nm sub-interval from 200 to 320nm as done for the GA3 spectral file. In band 2, a single k-term is calculated for each of the sub-intervals 320-400nm and 400-505nm. This allows the incoming solar flux to be supplied on these finer wavelength bands for experiments concerning solar spectral variability.

Absorption due to Sulphur dioxide (SO2, principally in the UV, plus near-IR) and Carbonyl sulphide (OCS, near-IR) is included based on HITRAN 2012 (only used for particular experimental configurations).

Total of 41 major gas k-terms.


\subsection*{Spectral file: sp\_lw\_ga7}

Sections are identical to sp\_lw\_ga3\_0 except for changes to gaseous absorption, thermal emission:

\subsubsection*{Spectral bands}

The nine spectral bands are identical to sp\_lw\_ga3\_0. Band limits are:

\begin{tabular}{l|l|l}
Band & Wavenumber (cm$^{-1}$) & Wavelength ($\mu$m)\\ \hline
1 & 1 - 400 & 25 - 10000 \\
2 & 400 - 550 & 18.18 - 25\\
3 & 550 - 590 and 750 - 800 & 12.5 - 13.33 and 16.95 - 18.18\\
4 & 590 - 750 & 13.33 - 16.95 \\
5 & 800 - 990 and 1120 - 1200 & 8.33 - 8.93 and 10.10 - 12.5\\
6 & 990 - 1120 & 8.93 - 10.10\\
7 & 1200 - 1330 & 7.52 - 8.33\\
8 & 1330 - 1500 & 6.67 - 7.52\\
9 & 1500 - 2995 & 3.34 - 6.67\\
\end{tabular}


\subsubsection*{Gaseous absorption}

Newly derived gaseous absorption for all gases (except CO2 in band 4) based on HITRAN 2012 and CAVIAR water vapour continuum. Scaling of absorption coefficients uses a look-up table of 59 pressures with 5 temperatures per pressure level based around a mid-latitude summer profile.

The improved representation of CO2 in the window region (more minor gas k-terms in bands 5 and 6) provides a better forcing response to increases in CO2 (tested up to x32 present day).

Greenhouse gases included: H2O, CO2, O3, N2O, CH4, CFC11, CFC12, CFC113, HCFC22 and HFC134a.

Absorption due to Sulphur dioxide (SO2) and Carbonyl sulphide (OCS) is included based on HITRAN 2012 (only used for particular experimental configurations).

Total of 81 major gas k-terms. The new method of ``hybrid'' scattering may be used with this spectral file. This will run the full scattering solver for 27 of the major gas k-terms (where their nominal optical depth is less than 10 in a mid-latitude summer atmosphere). The remaining 54 k-terms (optical depth $> 10$) will use a much cheaper non-scattering solver.

\subsubsection*{Thermal emission}

The Planckian function in each band is represented by a quartic fit in the temperature, generated by a least squares fit over the range 160 to 330 K. This increases the lower bound of the fit from 150K used with sp\_lw\_ga3\_0 and slightly improves the fit over the important temperature range for the Earth's atmosphere.

\subsection{Global Atmosphere Configuration 3}

\subsection*{Spectral file: sp\_sw\_ga3\_0 / spec\_sw\_ga3\_0}

Sections are identical to spec3a\_sw\_hadgem1\_5o\_rlfx except for changes to the solar spectrum (including Rayleigh coefficients), gaseous absorption, aerosols, and ice crystals:

\subsubsection*{Solar spectrum}

The Lean (2000, updated) spectrum \citep{Lean00} is based on satellite observations at wavelengths shorter than 735nm with the Kurucz spectrum at longer wavelengths. The satellite observations provided monthly data which have been meaned over the last 2 solar cycles (between 1983 and 2004 inclusive).


\subsubsection*{Gaseous absorption}

Changes to O3 k-terms in bands 1-3. The revision was made in order to improve ozone heating rate calculations and better incorporate solar variability. Briefly, the first UV band is divided into six relatively narrow sub-bands, each of which has only one ozone absorption coefficient so that, although the total number of bands is increased, the computational demands are similar to the previous k-distribution method for the UV band. Each new sub-band has physically realistic band limits and the ozone absorption coefficients are obtained from mean transmission functions calculated with high resolution (1 cm$^{-1}$) and using a fitting procedures similar to that described by \citet{Chou96}. Ozone cross-sections used in the calculations are a combination of Hitran2004 (0.24 - 0.34 $\mu$m), \citet{Molina86} (0.2 - 0.24 $\mu$m) and \citet{Voigt01} (above 0.34 $\mu$m). This new broad-band model has greater accuracy due to the higher number of bands within the UV. It also provides an easier vehicle for experiments in which variations in the solar irradiance spectrum may be imposed because the k-distribution of lines is restricted to the narrower bands.

For more information see \citet{Wenyi08}.


\subsubsection*{Aerosols}
 
Addition of 4 aerosol species: Fresh and Aged OCFF (Organic Carbon Fossil Fuel), `delta' aerosol, and nitrate aerosol. The optical properties of the 6 divisions of mineral dust have been revised using the set of refractive indices from \citet{Balkanski07}. This makes mineral dust less absorbing in the SW and more absorbing in the LW.


\subsubsection*{Ice crystals}

A new parametrisation for the optical properties of ice crystals has been developed by Anthony Baran based on the latest observed particle size distributions (from Paul Field) and an ensemble model of ice crystal type and orientation. The optical properties are linked directly to temperature and ice water content with no intermediate dependence on ice-crystal size. This parametrisation is added as type 9.


\subsection*{Spectral file: sp\_lw\_ga3\_0 / spec\_lw\_ga3\_0}

spec\_lw\_ga3\_0 is used for climate configurations and is favoured to spec\_lw\_ga3\_1 principally where a more accurate treatment of the stratosphere is required. Sections are identical to spec3a\_lw\_hadgem1\_5C except for changes to gaseous absorption, thermal emission, aerosols, and ice crystals (also note that obsolete coefficients for the continuum in band 9 have been removed):


\subsubsection*{Gaseous absorption}

New k-terms have been provided by Wenyi Zhong for CO2 in band 4 and O3 in band 6. This increases the total number of k-terms by 14 (with a corresponding increase in computational cost) but allows for a more accurate treatment of stratospheric absorption. The k-terms for CO2 and O3 have been taken from a spectral file developed by Wenyi Zhong for the ``Met Office Middle Atmosphere'' model based on HadCM3 \citep{Zhong2000}.


\subsubsection*{Thermal emission}

The Planckian function in each band is represented by a quartic fit in the temperature, generated by a least squares fit over the range 150 to 330 K. The previous fit (using the range of 180 to 330 K) for spec3a\_lw\_hadgem1\_5C could give negative emission for the very cold temperatures sometimes seen at the top of the model. 


\subsubsection*{Aerosols}
 
Addition of 4 aerosol species: Fresh and Aged OCFF (Organic Carbon Fossil Fuel), `delta' aerosol, and nitrate aerosol. The optical properties of the 6 divisions of mineral dust have been revised using the set of refractive indices from \citet{Balkanski07}. This makes mineral dust less absorbing in the SW and more absorbing in the LW.
Aerosol Optical Depth coefficients have been altered accordingly.


\subsubsection*{Ice crystals}
 
A new parametrisation (type 9) has been added to be used in conjunction with the new SW ice properties. These are based on the near-IR properties from spec\_sw\_ga3\_0 as the full LW properties have not yet been modelled.


\subsection*{Spectral file: sp\_lw\_ga3\_1 / spec\_lw\_ga3\_1}

spec\_lw\_ga3\_1 is used for forecast configurations and is favoured to spec\_lw\_ga3\_0 where speed of computation and more accurate treatment of the troposphere is required. Sections are identical to spec3a\_lw\_hadgem1\_5C except for changes to aerosols and ice crystals (also note that obsolete coefficients for the continuum in band 9 have been removed):


\subsubsection*{Aerosols}
 
Addition of 4 aerosol species: Fresh and Aged OCFF (Organic Carbon Fossil Fuel), `delta' aerosol, and nitrate aerosol. The optical properties of the 6 divisions of mineral dust have been revised using the set of refractive indices from \citet{Balkanski07}. This makes mineral dust less absorbing in the SW and more absorbing in the LW.
Aerosol Optical Depth coefficients have been altered accordingly.


\subsubsection*{Ice crystals}
 
A new parametrisation (type 9) has been added to be used in conjunction with the new SW ice properties. These are based on the near-IR properties from spec\_sw\_ga3\_0 as the full LW properties have not yet been modelled.


\subsection*{Spectral files: sp\_sw\_cloud3\_0 \& sp\_lw\_cloud3\_0}

These are simple spectral files designed specifically for use with the ``incremental radiative time-stepping'' scheme for improved sampling of cloud. They represent regions of high transmissivity in the SW and LW in order to capture the radiative effects of changes in low cloud. A full description of these files is available in \citet{Manners09}. (Namelist versions starting spec\_ are also available.)

\subsection{HadGEM2}

\subsection*{Spectral file: spec3a\_sw\_hadgem1\_5o\_rlfx}

spec3a\_sw\_hadgem1\_5o\_rlfx is used in the HadGEM2-A model and the global forecast model from PS20. All sections are identical to spec3a\_sw\_hadgem1\_3 except for changes to aerosols and Rayleigh scattering:


\subsubsection*{Rayleigh scattering bug-fix}

Rayleigh scattering coefficients in spec3a\_sw\_hadgem1\_3 were found to be in error by approximately 20\% due to a bug in the generating code. These are corrected here. Further information on this error and its impact in the global model is available in \citet{Haywood2008}.


\subsubsection*{Aerosols}

Mie scattering calculations have provided the optical properties for 7 additional aerosols: 6 size bins (also termed divisions) for mineral dust, and 1 mode representing biogenic aerosols from terpene emissions. The biogenic aerosol size distribution is lognormal, with a modal radius of 0.095 microns and a standard deviation of 1.5. Its density is 1300 kg/m3. Biogenic aerosols experience hygroscopic growth. In addition, parametrisations of Aitken Sulphate, Fresh Biomass (mode 1), and Aged Biomass (mode 2) have been changed. Aitken sulphate lognormal size distribution has now a modal radius of 0.0065 microns with a standard deviation of 1.3. Biomass aerosols are now hygroscopic following aircraft measurements. Their size distributions now use a modal radius of 0.1 and 0.12 microns for fresh and aged biomass, respectively, with a standard deviation of 1.3. Biomass aerosol density is now 1350 kg/m3. 


\subsection*{Spectral file: spec3a\_lw\_hadgem1\_5C}

spec3a\_lw\_hadgem1\_5C is used in the HadGEM2-A model and the global forecast model from PS20. All sections are identical to spec3a\_lw\_hadgem1\_3 except for changes to aerosols:


\subsubsection*{Aerosols}

Mie scattering calculations have provided the optical properties for 7 additional aerosols: 6 size bins (also termed divisions) for mineral dust, and 1 mode representing biogenic aerosols from terpene emissions. The biogenic aerosol size distribution is lognormal, with a modal radius of 0.095 microns and a standard deviation of 1.5. Its density is 1300 kg/m3. Biogenic aerosols experience hygroscopic growth. In addition, parametrisations of Aitken Sulphate, Fresh Biomass (mode 1), and Aged Biomass (mode 2) have been changed. Aitken sulphate lognormal size distribution has now a modal radius of 0.0065 microns with a standard deviation of 1.3. Biomass aerosols are now hygroscopic following aircraft measurements. Their size distributions now use a modal radius of 0.1 and 0.12 microns for fresh and aged biomass, respectively, with a standard deviation of 1.3. Biomass aerosol density is now 1350 kg/m3.

In addition, a new block, number 15, is introduced. It contains the specific absorption and scattering coefficients of each aerosol mode (in the same order as in the aerosol block 11). In contrast to the content of block 11, which are averaged across spectral bands, block-15 coefficients are monochromatic (given at specific wavelengths). They are used by the model to compute the aerosol optical depth at these wavelengths. There are 6 wavelengths, in the order: 0.38, 0.44, 0.55, 0.67, 0.87, 1.02 microns. As in the aerosol block 11, those aerosols which are hygroscopic have relative-humidity-dependent coefficients. 

\subsection{HadGEM1}

\subsection*{Spectral file: spec3a\_sw\_hadgem1\_3}

spec3a\_sw\_hadgem1\_3 is the standard SW spectral file
for HadGEM1.

The spectrum is divided into six bands, the
second and third of which are not true bands, as discussed above
under the remarks on block 1.

The solar spectrum is based on that published by \citet{Kurucz95}, and
this is used in the frequency weighting of Rayleigh scattering 
coefficients.

\subsubsection*{Gaseous absorption}

Gaseous absorption by water vapour, ozone, carbon dioxide and oxygen 
is included. Version 2.4 of the CKD continuum is included. The
foreign component is combined with the line data and fitted as one
entity. The self-broadened continuum is represented explicitly.
The spectroscopic data used in generating the absorption data come from 
HITRAN2000, with the published corrections, augmented by theoretical 
weak lines and extra observations from ESA (see \citet{Zhong2001}
for an introduction to this matter and further references). 
The data for gases other than water vapour are identical to those
used in HadCM3, as described in \citet{Cusack99ck}.

\subsubsection*{Aerosols}

Aerosols included comprise the five aerosols of the standard climatology
(\citet{Cusack98a}) and two modes each for sulphate, black carbon,
sea-salt and biomass aerosols and six divisions of dust aerosol.
The properties of aerosols depend on their nature and the size distribution.
Size distributions and optical properties for the climatological aerosols
are specified as in the standard WMO report (see \citet{Cusack98a} for
details). 
The single scattering parameters for aerosols are generated by running a 
Mie scattering code and averaging over the assumed size distribution.
The climatology is specified in terms of an optical depth, but densities
for the aerosols are not required or specified. However, the radiation
code works in terms of mass extinction coefficients, so a density must
be assumed. Provided that the same density is used in the code and in
the generation of the spectral file, its value is irrelevant and a 
conventional density of 1000 kgm$^{-3}$ has been assumed. If spectral
data for the climatological aerosols are combined with mass-loadings
specified other than through the climatology, it is necessary to
consider whether this density is appropriate. Note that climatological
aerosols will not be a part of the final standard version of HadGEM1, 
having been superceded by prognostics aerosols

Sulphate aerosols are hygroscopic, so their optical properties depend on
the relative humidity. The nature of this dependence is a matter for
aerosol modellers. From the point of view of generating radiative data,
a size distribution of the dry aerosol must be assumed. Two distinct
modes of aerosol are included in this file: the Aitken and accumulation
modes. For each of these modes, a log-normal size distribution is assumed.
For the Aitken mode, the modal radius, $\hat r = 24$ nm and the 
standard deviation $\sigma=1.45$. In the case of the accumulation mode
$\hat r = 95$ nm and $\sigma=1.4$. The density of dry aerosol is taken
as 1769 kgm${}^{-2}$.

Black carbon aerosols not not hygroscopic. They are represented as fresh
and aged aerosols, each obeying a log-normal distribution. In this case
$\hat r = 40$ nm and $\sigma=2.0$ for both modes. The density is taken
as 1000 kgm${}^{-2}$. (Note that these size distributions differ from
those used in {\tt spec3a\_sw\_3\_asol2c\_hadcm3}).

Film and jet modes of sea-salt aerosol are included. Data were generated 
using log-normal size distributions with $\hat r=0.1$ $\mu$m and
$\sigma=2.0$ for the film mode and $\hat r=1.0$ $\mu$m and $\sigma=2.0$
for the jet mode.

Prognostic dust aerosols are modelled using six size classes with limits as
follows: 6.32456E-8 -- 2.0E-7 (m), 2.0E-7 -- 6.32456E-7, 
6.32456E-7 -- 2.0E-6, 2.0E-6 -- 6.32456E-6, 6.32456E-6 -- 2.0E-5 and
2.0E-5 -- 6.32456E-5. The size distribution is taken as uniform within
each bin. The density of dust is taken as 2650 kgm${}^{-3}$. Data were
generated using a Mie scattering code, taking the refractive indices
given by \citet{Deepak83}.

Two modes of biomass smoke are included. For the fresh smoke (biomass 1),
a log-normal distribution with $\hat r=69$ $\mu$m and $\sigma=1.65$ is
assumed. For the aged smoke (biomass 2), $\hat r=200$ $\mu$m and $\sigma=1.58$.

For further details about aerosols, the documentation on this area should
be consulted.

\subsubsection*{Cloud droplets}

Data for water droplets were generated using a Mie scattering code. 
Whilst a single size distribution may be assumed for each species
of aerosol individually, size distributions for droplets vary widely,
depending on the location and moisture content of the atmosphere. Some
appropriate but variable measure of the size of droplet is required.
For radiative purposes, the
appropriate measure of size is the effective radius, $r_e$. $r_e$
may be parametrized or imposed (see section~\ref{sec:twostream}).
The numbers in the
spectral file represent coefficients in a parametrization. They
are generated by running a Mie scattering code for a number of different
size distributions at a range of wavelengths, averaging the single
scattering properties across the spectral bands, weighting with an
appropriate function of frequency and then fitting using some appropriate
function of the effective radius. This may clearly be done in many
different ways, and to allow general freedom, the concept of a {\em
type} of droplet is introduced. In the current file, four types are available,
namely 2, 3, 4 and 5. In all cases the size distributions specified by 
\citet{Rockel91} with effective
radii in the range 1.5 -- 50 microns were used as the 
basis of the Mie calculations. In particular, this uses a modified gamma distribution of this form:

\begin{equation}
\frac{{\rm d}n}{{\rm d}r} = \frac{N \beta \left(\frac{r}{r_m}\right)^{\alpha-1} e^{-\left(\frac{r}{r_m}\right)^{\beta}}}{r_m \Gamma \frac{\alpha}{\beta}}
\end{equation}

\smallskip
\noindent with parameters: $\alpha = (1/N_e) - 2$, $\beta = 1$, $r_m = N_e R_e$ 1.0E-6. Where the distribution variance ($N_e$) takes a number of values: $N_e = 0.01$, 0.1, 0.175, 0.25.

Calculations are done for each $N_e$ value at a number of different effective radii ($R_e$) and then a fit is made for each of the optical properties (extinction, single scattering albedo, asymmetry) against effective radius.

Weighting
was carried out using the solar spectrum of \citet{Labs70}. In the case of
types 2 and 4, the method of thin averaging (\citet{Edwards96rc}) was used,
whereas for types 3 and 5, the method of thick averaging was used. 
Types 2 and 3 have in fact been retained for historical consistency with
HadAM3 and are based on the linear fits of the functional form of
\citet{Slingo82}. Simple linear fits do not allow the use of a wide range
of particles sizes, as was required for use with the wider range of
studies of the indirect effects of sulphate envisaged with HadAM4.
To meet this need, new Pad\'e fits were developed, and their use is 
recommended. Type 4 corresponds to thin averaging and type 5 to thick
averaging. The use of type 5 is preferred for both convective and large-scale
clouds.

\subsubsection*{Ice crystals}

The generation of single-scattering data for ice crystals is more
complicated than for water clouds, because issues of crystal shape
must be addressed. When HadAM3 was defined, methods for generating 
single-scattering data for non-spherical particles were not available,
so data for ice particles were generated analogously to the approach
for water droplets, using the size distributions
for ice particles given by \citet{Rockel91} with effective radii
in the range 24 -- 80 microns, 
weighting with the solar spectrum of \citet{Labs70}
and using thin averaging for type 2 and thick averaging for type 3. In 
this case, because large-scale ice cloud is often thin, we recommend the
use of type 2 for large-scale cloud, but type 3 for convective cloud.
Since the definition of HadAM3, progress has been made with the treatment
of non-spherical particles. Type 7 invokes a treatment of ice  crystals
as planar polycrystals, based on the anomalous diffraction approximation
(see \citet{Kristjansson99} and \citet{Kristjansson00}). 
In this case, the
parameters represent a fit in terms of the mean maximum dimension of
the crystals. The mean maximum dimension is predicted in the model. At
releases up to 5.5, the use of this ice scheme automatically selects this
method of specifying the crystal size. 
A new parametrization \citep{Edwards07} was introduced for ice crystals at 5.5. This is based
on the representation of ice aggregates introduced by \citet{Baran01}.
The data were generated from the aggregate database (A. J. Baran pres. comm.)
using 83 representative size distributions measured during CEPEX and fitted
using the appropriate functional form. Thin averaging was performed. Note
that this fit is provided in terms of the effective dimension. It may be
selected by choosing type 8 for ice crystals.
Thickly averaged data are not
available for ice crystals.
({\it Technical Note: \citet{Kristjansson00} use tenth-order polynomial
fits to the optical properties, but the parametrization
in this file is based on two
splined quartic fits. The two fits are to the same data, but the tenth
order scheme was used in the paper for the convenience of running a
common scheme in CCM3 and the UM: the splined quartic fit had already
become part of HadAM4 when the tenth-order fit was developed.}) 

\subsection*{Spectral file: spec3a\_lw\_hadgem1\_3}

spec3a\_lw\_hadgem1\_3 is the standard longwave spectral file 
for HadGEM1. The spectrum is divided into nine bands, the
third and fifth of which are split, as discussed above
under the remarks on block 14. 

The Planckian function in each band is represented by a quartic fit
in the temperature, generated by a least squares fit over the range
180 to 330 K.

\subsubsection*{Gaseous absorption}

Gaseous absorption by water vapour, ozone, carbon dioxide, methane, 
nitrous oxide, CFC11, CFC12, CFC113, HCFC22, HFC125 and HFC134a is 
included. The spectroscopic data used in generating the absorption 
coefficients for gases other than the halocarbons and water vapour 
come from HITRAN92: 
for further details see \citet{Cusack99ck}. Absorption
cross-sections for the halocarbons were based on data supplied by
K. Shine (pers. comm.).
Data for absorption by water vapour were generated from HITRAN2000.
The water vapour continuum is represented 
using version 2.4 of the CKD model. The self-broadened continuum is
represented explicitly, while the foreign broadened continuum is
combined with the line absorption, the combined absorption being
fitted as if it were line data.

\subsubsection*{Aerosols}

Aerosols included comprise the five aerosols of the standard climatology
(\citet{Cusack98a}), two modes each for sulphate, black carbon,
sea-salt and biomass aerosols and six divisions of dust aerosol.
The properties of aerosols depend on their nature and the size distribution.
Size distributions and optical properties for the climatological aerosols
are specified as in the standard WMO report (see \citet{Cusack98a} for
details). 
The single scattering parameters for aerosols are generated by running a 
Mie scattering code and averaging over the assumed size distribution.
The climatology is specified in terms of an optical depth, but densities
for the aerosols are not required or specified. However, the radiation
code works in terms of mass extinction coefficients, so a density must
be assumed. Provided that the same density is used in the code and in
the generation of the spectral file, its value is irrelevant and a 
conventional density of 1000 kgm${}^{-3}$ has been assumed. If spectral
data for the climatological aerosols are combined with mass-loadings
specified other than through the climatology, it is necessary to
consider whether this density is appropriate. Note that climatological
aerosols are not included in the standard version of HadGEM1, being
replaced by prognostic aerosols.

Sulphate aerosols are hygroscopic, so their optical properties depend on
the relative humidity. The nature of this dependence is a matter for
aerosol modellers. From the point of view of generating radiative data,
a size distribution of the dry aerosol must be assumed. Two distinct
modes of aerosol are included in this file: the Aitken and accumulation
modes. For each of these modes, a log-normal size distribution is assumed.
For the Aitken mode, the modal radius, $\hat r = 24$ nm and the 
standard deviation $\sigma=1.45$. In the case of the accumulation mode
$\hat r = 95$ nm and $\sigma=1.4$. The density of dry aerosol is taken
as 1769 kgm${}^{-2}$.

Black carbon aerosols not not hygroscopic. They are represented as fresh
and aged aerosols, each obeying a log-normal distribution. In this case
$\hat r = 40$ nm and $\sigma=2.0$ for the both the fresh and modes.

Film and jet modes of sea-salt aerosol are included. Data were generated 
for log-normal size distributions with $\hat r=0.1$ $\mu$m and
$\sigma=2.0$ for the film mode and $\hat r=1.0$ $\mu$m and $\sigma=2.0$
for the jet mode.

Prognostic dust aerosols are modelled using six size classes with limits as
follows: 6.32456E-8 -- 2.0E-7 (m), 2.0E-7 -- 6.32456E-7, 
6.32456E-7 -- 2.0E-6, 2.0E-6 -- 6.32456E-6, 6.32456E-6 -- 2.0E-5 and
2.0E-5 -- 6.32456E-5. The size distribution is taken as uniform within
each bin. The density of dust is taken as 2650 kgm${}^{-3}$. Data were
generated using a Mie scattering code, taking the refractive indices
given by \citet{Deepak83}.

Two modes of biomass smoke are included. For the fresh smoke (biomass 1),
a log-normal distribution with $\hat r=69$ $\mu$m and $\sigma=1.65$ is
assumed. For the aged smoke (biomass 2), $\hat r=200$ $\mu$m and $\sigma=1.58$.

For further details about aerosols, the documentation on this area should
be consulted.

\subsubsection*{Cloud droplets}

Data for water droplets were generated using a Mie scattering code. 
Whilst a single size distribution may be assumed for each species
of aerosol individually, size distributions for droplets vary widely,
depending on the location and moisture content of the atmosphere. Some
appropriate but variable measure of the size of droplet is required.
For radiative purposes, the
appropriate measure of size is the effective radius, $r_e$. $r_e$
may be parametrized or imposed (see section~\ref{sec:twostream}). The numbers in the
spectral file represent coefficients in a parametrization. They
are generated by running a Mie scattering code for a number of different
size distributions at a range of wavelengths, averaging the single
scattering properties across the spectral bands, weighting with an
appropriate function of frequency and then fitting using some appropriate
function of the effective radius. This may clearly be done in many
different ways, and to allow general freedom, the concept of a {\em
type} of droplet is introduced.  Data for type 1 were obtained by
using the size distributions specified by \citet{Rockel91} with effective
radii in the range 1.5 -- 50 microns 
as the basis of the Mie calculations. Weighting
was carried out using a Planckian function at a temperature of 250 K,
and spectral averaging was carried out using the method of 
thin averaging (\citet{Edwards96rc}) and
the functional form of \citet{Slingo82} was used for fitting. 
These data are retained for historical consistency and the use of
the Pad\'e fits of types 4 and 5, which are valid over a wider range
of effective radii is now recommended. These data were generated from 
the same sources as type 1, but differ in the fitting used. Type 4
was generated using thin averaging and type 5 with thick averaging.

\subsubsection*{Ice crystals}

The generation of single-scattering data for ice crystals is more
complicated than for water clouds, because issues of crystal shape
must be addressed. When HadAM3 was defined, methods for generating 
single-scattering data for non-spherical particles were not available,
so data for ice particles were generated analogously to the approach
for water droplets, using the size distributions for ice particles 
given by \citet{Rockel91}
with effective radii in the range 24 -- 80 microns, 
weighting with the a Planckian 
function at 250 K and using thin averaging. The functional form of 
\citet{Slingo82} was used again: only data for type 1 were initially
available.
Since the definition of HadAM3, progress has been made with the treatment
of non-spherical particles. Type 7 invokes a treatment of ice crystals
as planar polycrystals, based on the anomalous diffraction approximation
(see \citet{Kristjansson99} and \citet{Kristjansson00}). 
In this case, the
parameters represent a fit in terms of the mean maximum dimension of
the crystals. The mean maximum dimension is predicted in the model. At
releases up to 5.5, the use of this ice scheme automatically selects this
method of specifying the crystal size. 
A new parametrization \citep{Edwards07} was introduced for ice crystals at 5.5. This is based
on the representation of ice aggregates introduced by \citet{Baran03}.
The data were generated from the aggregate database (A. J. Baran pres. comm.)
using 83 representative size distributions measured during CEPEX and fitted
using the appropriate functional form. Thin averaging was performed. Note
that this fit is provided in terms of the effective dimension. It may be
selected by choosing type 8 for ice crystals.
Thickly averaged data are not
available for non-spherical ice crystals.
({\it Technical Note: \citet{Kristjansson00} use tenth-order polynomial
fits to the optical properties, but the parametrization
in this file is based on two
splined quartic fits. The two fits are to the same data, but the tenth
order scheme was used in the paper for the convenience of running a
common scheme in CCM3 and the UM: the splined quartic fit had already
become part of HadAM4 when the tenth-order fit was developed.}) 


\subsection{Older spectral files}

\begin{enumerate}

\item
{\large \tt spec3a\_sw\_3\_asol2c\_hadcm3} is the standard shortwave spectral 
file used in HadCM3 runs. The spectrum is divided into six bands, the
second and third of which are not true bands, as discussed above
under the remarks on block 1.

The solar spectrum is based on that published by \citet{Labs70}, and
this is used in the frequency weighting of Rayleigh scattering 
coefficients.

Gaseous absorption by water vapour (without the continuum), ozone, 
carbon dioxide and oxygen is included. The
spectroscopic data used in generating the absorption data come from 
HITRAN92, except
for the data on ozone which were generated from LOWTRAN7: for further
details see \citet{Cusack99ck}.

Aerosols included comprise the five aerosols of the standard climatology
(\citet{Cusack98a}), two modes of sulphate aerosol and two modes of
black carbon aerosol.
The properties of aerosols depend on their nature and the size distribution.
Size distributions and optical properties for the climatological aerosols
are specified as in the standard WMO report (see \citet{Cusack98a} for
details). 
The single scattering parameters for aerosols are generated by running a 
Mie scattering code and averaging over the assumed size distribution.
The climatology is specified in terms of an optical depth, but densities
for the aerosols are not required or specified. However, the radiation
code works in terms of mass extinction coefficients, so a density must
be assumed. Provided that the same density is used in the code and in
the generation of the spectral file, its value is irrelevant and a 
conventional density of 1000 kgm${}^{-3}$ has been assumed. If spectral
data for the climatological aerosols are combined with mass-loadings
specified other than through the climatology, it is necessary to
consider whether this density is appropriate.

Sulphate aerosols are hygroscopic, so their optical properties depend on
the relative humidity. The nature of this dependence is a matter for
aerosol modellers. From the point of view of generating radiative data,
a size distribution of the dry aerosol must be assumed. Two distinct
modes of aerosol are included in this file: the Aitken and accumulation
modes. For each of these modes, a log-normal size distribution is assumed.
For the Aitken mode, the modal radius, $\hat r = 24$ nm and the 
standard deviation $\sigma=1.45$. In the case of the accumulation mode
$\hat r = 95$ nm and $\sigma=1.4$. The density of dry aerosol is taken
as 1769 kgm${}^{-2}$.

Black carbon aerosols not not hygroscopic. They are represented as fresh
and aged aerosols, each obeying a log-normal distribution. In this case
$\hat r = 20$ nm and $\sigma=2.0$ for the fresh modes, 
but $\hat r = 100$ nm and $\sigma=2.0$ for the aged mode. 
The density is taken as 1000 kgm${}^{-2}$.

Data for water droplets were generated using a Mie scattering code. 
Whilst a single size distribution may be assumed for each species
of aerosol individually, size distributions for droplets vary widely,
depending on the location and moisture content of the atmosphere. Some
appropriate but variable measure of the size of droplet is required.
For radiative purposes, the
appropriate measure of size is the effective radius, $r_e$. $r_e$
may be parametrized or imposed (see section~\ref{sec:twostream}). The numbers in the
spectral file represent coefficients in a parametrization. They
are generated by running a Mie scattering code for a number of different
size distributions at a range of wavelengths, averaging the single
scattering properties across the spectral bands, weighting with an
appropriate function of frequency and then fitting using some appropriate
function of the effective radius. This may clearly be done in many
different ways, and to allow general freedom, the concept of a {\em
type} of droplet is introduced. In the current file, two types are available,
namely 2 and 3. In both cases the size distributions specified by 
\citet{Rockel91} with effective
radii in the range 1.5 -- 50 microns were used as 
the basis of the Mie calculations. Weighting
was carried out using the solar spectrum of \citet{Labs70}. In the case of
type 2, the method of thin averaging (\citet{Edwards96rc}) was used,
whereas for type 3, the method of thick averaging was used. In both cases
the functional form of \citet{Slingo82} was used. Tests against more
highly spectrally resolved data suggest that thick averaging is more 
representative for water clouds, and the use of type 3 for both large-scale
and convective clouds is recommended.

The generation of single-scattering data for ice crystals is more
complicated than for water clouds, because issues of crystal shape
must be addressed. When HadAM3 was defined, methods for generating 
single-scattering data for non-spherical particles were not available,
so data for ice particles were generated analogously to the approach
for water droplets, using the size distributions
for ice particles given by \citet{Rockel91} with effective radii
in the range 24 -- 80 microns, 
weighting with the solar spectrum of \citet{Labs70}
and using thin averaging for type 2 and thick averaging for type 3. In 
this case, because large-scale ice cloud is often thin, we recommend the
use of type 2 for large-scale cloud, but type 3 for convective cloud.
Since the definition of HadAM3, progress has been made with the treatment
of non-spherical particles. Type 7 invokes a treatment of ice  crystals
as planar polycrystals, based on the anomalous diffraction approximation
(see \citet{Kristjansson99} and \citet{Kristjansson00}). 
In this case, the
parameters represent a fit in terms of the mean maximum dimension of
the crystals. The mean maximum dimension is predicted in the model. At
releases up to 5.5, the use of this ice scheme automatically selects this
method of specifying the crystal size. Thickly averaged data are not
available for non-spherical ice crystals.
({\it Technical Note: \citet{Kristjansson00} use tenth-order polynomial
fits to the optical properties, but the parametrization
in this file is based on two
splined quartic fits. The two fits are to the same data, but the tenth
order scheme was used in the paper for the convenience of running a
common scheme in CCM3 and the UM: the splined quartic fit had already
become part of HadAM4 when the tenth-order fit was developed.}) 


\item
{\large \tt spec3a\_sw\_hadcm4} is the standard shortwave spectral 
file used in HadAM4 runs. The spectrum is divided into six bands, the
second and third of which are not true bands, as discussed above
under the remarks on block 1.

The solar spectrum is based on that published by \citet{Labs70}, and
this is used in the frequency weighting of Rayleigh scattering 
coefficients.

Gaseous absorption by water vapour, ozone, carbon dioxide and oxygen 
is included. Version 2.1 of the CKD continuum is included. The
foreign component is combined with the line data and fitted as one
entity. The self-broadened continuum is represented explicitly.
The spectroscopic data used in generating the absorption data come from 
HITRAN92. Absorption by water vapour in the near infra-red has been
improved relative to the treatment in {\tt spec3a\_sw\_asol2c\_hadcm4},
as used in HadCM3, by the addition of an extra $k$-term in the fourth
band.  The data for gases other than water vapour are identical to those
used in HadCM3, as described in \citet{Cusack99ck}.

Aerosols included comprise the five aerosols of the standard climatology
(\citet{Cusack98a}), and two modes of sulphate aerosols.
The properties of aerosols depend on their nature and the size distribution.
Size distributions and optical properties for the climatological aerosols
are specified as in the standard WMO report (see \citet{Cusack98a} for
details). 
The single scattering parameters for aerosols are generated by running a 
Mie scattering code and averaging over the assumed size distribution.
The climatology is specified in terms of an optical depth, but densities
for the aerosols are not required or specified. However, the radiation
code works in terms of mass extinction coefficients, so a density must
be assumed. Provided that the same density is used in the code and in
the generation of the spectral file, its value is irrelevant and a 
conventional density of 1000 kgm${}^{-3}$ has been assumed. If spectral
data for the climatological aerosols are combined with mass-loadings
specified other than through the climatology, it is necessary to
consider whether this density is appropriate.

Sulphate aerosols are hygroscopic, so their optical properties depend on
the relative humidity. The nature of this dependence is a matter for
aerosol modellers. From the point of view of generating radiative data,
a size distribution of the dry aerosol must be assumed. Two distinct
modes of aerosol are included in this file: the Aitken and accumulation
modes. For each of these modes, a log-normal size distribution is assumed.
For the Aitken mode, the modal radius, $\hat r = 24$ nm and the 
standard deviation $\sigma=1.45$. In the case of the accumulation mode
$\hat r = 95$ nm and $\sigma=1.4$. The density of dry aerosol is taken
as 1769 kgm${}^{-2}$.

Data for water droplets were generated using a Mie scattering code. 
Whilst a single size distribution may be assumed for each species
of aerosol individually, size distributions for droplets vary widely,
depending on the location and moisture content of the atmosphere. Some
appropriate but variable measure of the size of droplet is required.
For radiative purposes, the
appropriate measure of size is the effective radius, $r_e$. $r_e$
may be parametrized or imposed (see section~\ref{sec:twostream}). The numbers in the
spectral file represent coefficients in a parametrization. They
are generated by running a Mie scattering code for a number of different
size distributions at a range of wavelengths, averaging the single
scattering properties across the spectral bands, weighting with an
appropriate function of frequency and then fitting using some appropriate
function of the effective radius. This may clearly be done in many
different ways, and to allow general freedom, the concept of a {\em
type} of droplet is introduced. In the current file, four types are available,
namely 2, 3, 4 and 5. In all cases the size distributions specified by 
\citet{Rockel91} with effective
radii in the range 1.5 -- 50 microns 
were used as the basis of the Mie calculations. Weighting
was carried out using the solar spectrum of \citet{Labs70}. In the case of
types 2 and 4, the method of thin averaging (\citet{Edwards96rc}) was used,
whereas for types 3 and 5, the method of thick averaging was used. 
Types 2 and 3 have in fact been retained for historical consistency with
HadAM3 and are based on the linear fits of the functional form of
\citet{Slingo82}. Simple linear fits do not allow the use of a wide range
of particles sizes, as was required for use with the wider range of
studies of the indirect effects of sulphate envisaged with HadAM4.
To meet this need, new Pad\'e fits were developed, and their use is 
recommended. Type 4 corresponds to thin averaging and type 5 to thick
averaging. The use of type 5 is preferred for both convective and large-scale
clouds.

The generation of single-scattering data for ice crystals is more
complicated than for water clouds, because issues of crystal shape
must be addressed. When HadAM3 was defined, methods for generating 
single-scattering data for non-spherical particles were not available,
so data for ice particles were generated analogously to the approach
for water droplets, using the size distributions
for ice particles given by \citet{Rockel91} with effective radii
in the range 24 -- 80 microns, 
weighting with the solar spectrum of \citet{Labs70}
and using thin averaging for type 2 and thick averaging for type 3. In 
this case, because large-scale ice cloud is often thin, we recommend the
use of type 2 for large-scale cloud, but type 3 for convective cloud.
Since the definition of HadAM3, progress has been made with the treatment
of non-spherical particles. Type 7 invokes a treatment of ice  crystals
as planar polycrystals, based on the anomalous diffraction approximation
(see \citet{Kristjansson99} and \citet{Kristjansson00}). 
In this case, the
parameters represent a fit in terms of the mean maximum dimension of
the crystals. The mean maximum dimension is predicted in the model. At
releases up to 5.5, the use of this ice scheme automatically selects this
method of specifying the crystal size. Thickly averaged data are not
available for ice crystals.
({\it Technical Note: \citet{Kristjansson00} use tenth-order polynomial
fits to the optical properties, but the parametrization
in this file is based on two
splined quartic fits. The two fits are to the same data, but the tenth
order scheme was used in the paper for the convenience of running a
common scheme in CCM3 and the UM: the splined quartic fit had already
become part of HadAM4 when the tenth-order fit was developed.}) 



\item
{\large \tt spec3a\_sw\_h4\_meso2} is a spectral file designed for use 
with the mesoscale mode. {\em Important Note: This file has been developed
for use where speed of execution is critical and the balance between
speed and accuracy is very much toward speed, with minimal numbers of
$k$-terms being used for each gas. It is used operationally
only for mesoscale runs out to 36 hours and not for global or climate
runs. Its use for off-line radiation calculations is not encouraged.}

The shortwave spectral region is divided into five bands.
The solar spectrum is based on that published by \citet{Labs70}, and
this is used in the frequency weighting of Rayleigh scattering 
coefficients.

Gaseous absorption by water vapour, ozone and carbon dioxide
is included. Version 2.1 of the CKD continuum is included: here
the self and foreign components of the continuum are treated
separately. 
The spectroscopic data used in generating the absorption data come from 
HITRAN92.

Aerosols included comprise the five aerosols of the standard climatology
(\citet{Cusack98a}).
The properties of aerosols depend on their nature and the size distribution.
Size distributions and optical properties for the climatological aerosols
are specified as in the standard WMO report (see \citet{Cusack98a} for
details). 
The single scattering parameters for aerosols are generated by running a 
Mie scattering code and averaging over the assumed size distribution.
The climatology is specified in terms of an optical depth, but densities
for the aerosols are not required or specified. However, the radiation
code works in terms of mass extinction coefficients, so a density must
be assumed. Provided that the same density is used in the code and in
the generation of the spectral file, its value is irrelevant and a 
conventional density of 1000 kgm${}^{-3}$ has been assumed. If spectral
data for the climatological aerosols are combined with mass-loadings
specified other than through the climatology, it is necessary to
consider whether this density is appropriate.

Data for water droplets were generated using a Mie scattering code. 
Whilst a single size distribution may be assumed for each species
of aerosol individually, size distributions for droplets vary widely,
depending on the location and moisture content of the atmosphere. Some
appropriate but variable measure of the size of droplet is required.
For radiative purposes, the
appropriate measure of size is the effective radius, $r_e$. $r_e$
may be parametrized or imposed (see section~\ref{sec:twostream}). The numbers in the
spectral file represent coefficients in a parametrization. They
are generated by running a Mie scattering code for a number of different
size distributions at a range of wavelengths, averaging the single
scattering properties across the spectral bands, weighting with an
appropriate function of frequency and then fitting using some appropriate
function of the effective radius. This may clearly be done in many
different ways, and to allow general freedom, the concept of a {\em
type} of droplet is introduced. In the current file, four types are available,
namely 2, 3, 4 and 5. In all cases the size distributions specified by 
\citet{Rockel91} were used as the basis of the Mie calculations. Weighting
was carried out using the solar spectrum of \citet{Labs70}. In the case of
types 2 and 4, the method of thin averaging (\citet{Edwards96rc}) was used,
whereas for types 3 and 5, the method of thick averaging was used. 
Types 2 and 3 have in fact been retained for historical consistency with
HadAM3 and are based on the linear fits of the functional form of
\citet{Slingo82}. Simple linear fits do not allow the use of a wide range
of particles sizes, as was required for use with the wider range of
studies of the indirect effects of sulphate envisaged with HadAM4.
To meet this need, new Pad\'e fits were developed, and their use is 
recommended. Type 4 corresponds to thin averaging and type 5 to thick
averaging. The use of type 5 is preferred for both convective and large-scale
clouds.

The generation of single-scattering data for ice crystals is more
complicated than for water clouds, because issues of crystal shape
must be addressed. When HadAM3 was defined, methods for generating 
single-scattering data for non-spherical particles were not available,
so data for ice particles were generated analogously to the approach
for water droplets, using the size distributions
for ice particles given by \citet{Rockel91}
with effective radii in the range 24 -- 80 microns,
weighting with the solar spectrum of \citet{Labs70}
and using thin averaging for type 2 and thick averaging for type 3. In 
this case, because large-scale ice cloud is often thin, we recommend the
use of type 2 for large-scale cloud, but type 3 for convective cloud.
Since the definition of HadAM3, progress has been made with the treatment
of non-spherical particles. Type 7 invokes a treatment of ice  crystals
as planar polycrystals, based on the anomalous diffraction approximation
(see \citet{Kristjansson99} and \citet{Kristjansson00}). 
In this case, the
parameters represent a fit in terms of the mean maximum dimension of
the crystals. The mean maximum dimension is predicted in the model. At
releases up to 5.5, the use of this ice scheme automatically selects this
method of specifying the crystal size. Thickly averaged data are not
available for ice crystals.
({\it Technical Note: \citet{Kristjansson00} use tenth-order polynomial
fits to the optical properties, but the parametrization
in this file is based on two
splined quartic fits. The two fits are to the same data, but the tenth
order scheme was used in the paper for the convenience of running a
common scheme in CCM3 and the UM: the splined quartic fit had already
become part of HadAM4 when the tenth-order fit was developed.}) 


\item
{\large \tt spec3a\_lw\_3\_asol2c\_hadcm3} is the standard longwave spectral 
file used in HadCM3 runs. The spectrum is divided into eight bands, the
third and fifth of which are split, as discussed above
under the remarks on block 14.

The Planckian function in each band is represented by a quartic fit
in the temperature, generated by a least squares fit over the range
200 to 300 K.

Gaseous absorption by water vapour, ozone, carbon dioxide, methane, 
nitrous oxide, CFC11, CFC12, CFC113, HCFC22, HFC125 and HFC134a is 
included. The spectroscopic data used in generating the absorption 
parametrizations for gases other than halocarbons come from HITRAN92:
for further details see \citet{Cusack99ck}. Data for CFC11 and 
CFC12 are taken from \citet{Varanasi88}, while cross-sectional
data for other gases were supplied by K.~Shine (pers. comm.).
({\it Note in respect of potential revision of these data: 
Cross-sectional data are now included in the HITRAN 
database and the use of this source of data is recommended
for future work.})
The water vapour continuum is represented 
using version 2.1 of the CKD model. The self-broadened continuum is
represented explicitly, while the foreign broadened continuum is
combined with the line absorption, the combined absorption being
fitted as if it were line data.

Aerosols included comprise the five aerosols of the standard climatology
(\citet{Cusack98a}), two modes of sulphate aerosol and two modes of
black carbon aerosol.
The properties of aerosols depend on their nature and the size distribution.
Size distributions and optical properties for the climatological aerosols
are specified as in the standard WMO report (see \citet{Cusack98a} for
details). 
The single scattering parameters for aerosols are generated by running a 
Mie scattering code and averaging over the assumed size distribution.
The climatology is specified in terms of an optical depth, but densities
for the aerosols are not required or specified. However, the radiation
code works in terms of mass extinction coefficients, so a density must
be assumed. Provided that the same density is used in the code and in
the generation of the spectral file, its value is irrelevant and a 
conventional density of 1000 kgm${}^{-3}$ has been assumed. If spectral
data for the climatological aerosols are combined with mass-loadings
specified other than through the climatology, it is necessary to
consider whether this density is appropriate.

Sulphate aerosols are hygroscopic, so their optical properties depend on
the relative humidity. The nature of this dependence is a matter for
aerosol modellers. From the point of view of generating radiative data,
a size distribution of the dry aerosol must be assumed. Two distinct
modes of aerosol are included in this file: the Aitken and accumulation
modes. For each of these modes, a log-normal size distribution is assumed.
For the Aitken mode, the modal radius, $\hat r = 24$ nm and the 
standard deviation $\sigma=1.45$. In the case of the accumulation mode
$\hat r = 95$ nm and $\sigma=1.4$. The density of dry aerosol is taken
as 1769 kgm${}^{-2}$.

Black carbon aerosols not not hygroscopic. They are represented as fresh
and aged aerosols, each obeying a log-normal distribution. In this case
$\hat r = 20$ nm and $\sigma=2.0$ for the fresh modes, 
but $\hat r = 100$ nm and $\sigma=2.0$ for the aged mode. 
The density is taken as 1000 kgm${}^{-2}$.

Data for water droplets were generated using a Mie scattering code. 
Whilst a single size distribution may be assumed for each species
of aerosol individually, size distributions for droplets vary widely,
depending on the location and moisture content of the atmosphere. Some
appropriate but variable measure of the size of droplet is required.
For radiative purposes, the
appropriate measure of size is the effective radius, $r_e$. $r_e$
may be parametrized or imposed (see section~\ref{sec:twostream}). The numbers in the
spectral file represent coefficients in a parametrization. They
are generated by running a Mie scattering code for a number of different
size distributions at a range of wavelengths, averaging the single
scattering properties across the spectral bands, weighting with an
appropriate function of frequency and then fitting using some appropriate
function of the effective radius. This may clearly be done in many
different ways, and to allow general freedom, the concept of a {\em
type} of droplet is introduced. In this file, only one type is available,
type 1. The size distributions specified by 
\citet{Rockel91} with effective
radii in the range 1.5 -- 50 microns 
were used as the basis of the Mie calculations. Weighting
was carried out using a Planckian function at a temperature of 250 K,
and spectral averaging was carries out using the method of 
thin averaging (\citet{Edwards96rc}).
The functional form of \citet{Slingo82} was used for fitting. 

The generation of single-scattering data for ice crystals is more
complicated than for water clouds, because issues of crystal shape
must be addressed. When HadAM3 was defined, methods for generating 
single-scattering data for non-spherical particles were not available,
so data for ice particles were generated analogously to the approach
for water droplets, using the size distributions for ice particles 
given by \citet{Rockel91}
with effective radii in the range 24 -- 80 microns, 
weighting with the a Planckian 
function at 250 K and using thin averaging. The functional form of 
\citet{Slingo82} was used again: only data for type 1 were initially
available.
Since the definition of HadAM3, progress has been made with the treatment
of non-spherical particles. Type 7 invokes a treatment of ice  crystals
as planar polycrystals, based on the anomalous diffraction approximation
(see \citet{Kristjansson99} and \citet{Kristjansson00}). 
In this case, the
parameters represent a fit in terms of the mean maximum dimension of
the crystals. The mean maximum dimension is predicted in the model. At
releases up to 5.5, the use of this ice scheme automatically selects this
method of specifying the crystal size. Thickly averaged data are not
available for non-spherical ice crystals.
({\it Technical Note: \citet{Kristjansson00} use tenth-order polynomial
fits to the optical properties, but the parametrization
in this file is based on two
splined quartic fits. The two fits are to the same data, but the tenth
order scheme was used in the paper for the convenience of running a
common scheme in CCM3 and the UM: the splined quartic fit had already
become part of HadAM4 when the tenth-order fit was developed.}) 



\item
{\large \tt spec3a\_lw\_hadcm4\_N} is the standard longwave spectral 
file used in HadAM4 runs. The spectrum is divided into nine bands, the
third and fifth of which are split, as discussed above
under the remarks on block 14. This nine-band configuration was
developed from that used in HadAM3. In order to improve the treatment
of the spectral overlap between absorption by methane and nitrous
oxide in the region 1200 -- 1500 cm${}^{-1}$, it was decided to split
the original seventh spectral band, but as tuning of HadAM4 was well
advenced at that stage, it was desired not to alter the spectral 
characteristics of absorption by water vapour. Consequently, the
seventh and eighth bands are not true spectral bands (see the general
discussion of block 1 above).

The Planckian function in each band is represented by a quartic fit
in the temperature, generated by a least squares fit over the range
200 to 300 K, but the fits for bands 7 and 8 are a partitioning of
the fit for the region 1200 -- 1500 cm${}^{-1}$ for the reasons discussed
above.

Gaseous absorption by water vapour, ozone, carbon dioxide, methane, 
nitrous oxide, CFC11, CFC12, CFC113, HCFC22, HFC125 and HFC134a is 
included. The spectroscopic data used in generating the absorption 
coefficients for gases other than the halocarbons come from HITRAN92: 
for further details see \citet{Cusack99ck}. Absorption
cross-sections for the halocarbons were based on data supplied by
K. Shine (pers. comm.).
The water vapour continuum is represented 
using version 2.1 of the CKD model. The self-broadened continuum is
represented explicitly, while the foreign broadened continuum is
combined with the line absorption, the combined absorption being
fitted as if it were line data.

Aerosols included comprise the five aerosols of the standard climatology
(\citet{Cusack98a}) and two modes of sulphate aerosol.
The properties of aerosols depend on their nature and the size distribution.
Size distributions and optical properties for the climatological aerosols
are specified as in the standard WMO report (see \citet{Cusack98a} for
details). 
The single scattering parameters for aerosols are generated by running a 
Mie scattering code and averaging over the assumed size distribution.
The climatology is specified in terms of an optical depth, but densities
for the aerosols are not required or specified. However, the radiation
code works in terms of mass extinction coefficients, so a density must
be assumed. Provided that the same density is used in the code and in
the generation of the spectral file, its value is irrelevant and a 
conventional density of 1000 kgm${}^{-3}$ has been assumed. If spectral
data for the climatological aerosols are combined with mass-loadings
specified other than through the climatology, it is necessary to
consider whether this density is appropriate.

Sulphate aerosols are hygroscopic, so their optical properties depend on
the relative humidity. The nature of this dependence is a matter for
aerosol modellers. From the point of view of generating radiative data,
a size distribution of the dry aerosol must be assumed. Two distinct
modes of aerosol are included in this file: the Aitken and accumulation
modes. For each of these modes, a log-normal size distribution is assumed.
For the Aitken mode, the modal radius, $\hat r = 24$ nm and the 
standard deviation $\sigma=1.45$. In the case of the accumulation mode
$\hat r = 95$ nm and $\sigma=1.4$. The density of dry aerosol is taken
as 1769 kgm${}^{-2}$.

Data for water droplets were generated using a Mie scattering code. 
Whilst a single size distribution may be assumed for each species
of aerosol individually, size distributions for droplets vary widely,
depending on the location and moisture content of the atmosphere. Some
appropriate but variable measure of the size of droplet is required.
For radiative purposes, the
appropriate measure of size is the effective radius, $r_e$. $r_e$
may be parametrized or imposed (see section~\ref{sec:twostream}). The numbers in the
spectral file represent coefficients in a parametrization. They
are generated by running a Mie scattering code for a number of different
size distributions at a range of wavelengths, averaging the single
scattering properties across the spectral bands, weighting with an
appropriate function of frequency and then fitting using some appropriate
function of the effective radius. This may clearly be done in many
different ways, and to allow general freedom, the concept of a {\em
type} of droplet is introduced.  Data for type 1 were obtained by
using the size distributions specified by \citet{Rockel91} 
with effective radii in the range 1.5 -- 50 microns 
as the basis of the Mie calculations. Weighting
was carried out using a Planckian function at a temperature of 250 K,
and spectral averaging was carried out using the method of 
thin averaging (\citet{Edwards96rc}) and
the functional form of \citet{Slingo82} was used for fitting. 
These data are retained for historical consistency and the use of
the Pad\'e fits of types 4 and 5, which are valid over a wider range
of effective radii is now recommended. These data were generated from 
the same sources as type 1, but differ in the fitting used. Type 4
was generated using thin averaging and type 5 with thick averaging.

The generation of single-scattering data for ice crystals is more
complicated than for water clouds, because issues of crystal shape
must be addressed. When HadAM3 was defined, methods for generating 
single-scattering data for non-spherical particles were not available,
so data for ice particles were generated analogously to the approach
for water droplets, using the size distributions for ice particles 
given by \citet{Rockel91}
with effective radii in the range 24 -- 80 microns, 
weighting with the a Planckian 
function at 250 K and using thin averaging. The functional form of 
\citet{Slingo82} was used again: only data for type 1 were initially
available.
Since the definition of HadAM3, progress has been made with the treatment
of non-spherical particles. Type 7 invokes a treatment of ice crystals
as planar polycrystals, based on the anomalous diffraction approximation
(see \citet{Kristjansson99} and \citet{Kristjansson00}). 
In this case, the
parameters represent a fit in terms of the mean maximum dimension of
the crystals. The mean maximum dimension is predicted in the model. At
releases up to 5.5, the use of this ice scheme automatically selects this
method of specifying the crystal size. Thickly averaged data are not
available for non-spherical ice crystals.
({\it Technical Note: \citet{Kristjansson00} use tenth-order polynomial
fits to the optical properties, but the parametrization
in this file is based on two
splined quartic fits. The two fits are to the same data, but the tenth
order scheme was used in the paper for the convenience of running a
common scheme in CCM3 and the UM: the splined quartic fit had already
become part of HadAM4 when the tenth-order fit was developed.}) 



\item
{\large \tt spec3a\_lw\_h4\_meso2} is a spectral file designed for use 
with the mesoscale mode. {\em Important Note: This file has been developed
for use where speed of execution is critical and the balance between
speed and accuracy is very much toward speed, with minimal numbers of
$k$-terms being used for each gas. It is used operationally
only for mesoscale runs out to 36 hours and not for global or climate
runs. Its use for off-line radiation calculations is not encouraged.}

The longwave spectral region is divided into five bands.
The Planckian function in each band is represented by a quartic fit
in the temperature, generated by a least squares fit over the range
190 to 310 K.

Gaseous absorption by water vapour, ozone and carbon dioxide and nitrous
oxide is included. Version 2.1 of the CKD continuum is included. The
self-broadened component is represented explicitly, but the foreign-broadened
component has been added to the line data for water vapour and the $k$-terms
represent a fit to the combined entity.
The spectroscopic data used in generating the absorption data come from 
HITRAN92.

Aerosols included comprise the five aerosols of the standard climatology
(\citet{Cusack98a}).
The properties of aerosols depend on their nature and the size distribution.
Size distributions and optical properties for the climatological aerosols
are specified as in the standard WMO report (see \citet{Cusack98a} for
details). 
The single scattering parameters for aerosols are generated by running a 
Mie scattering code and averaging over the assumed size distribution.
The climatology is specified in terms of an optical depth, but densities
for the aerosols are not required or specified. However, the radiation
code works in terms of mass extinction coefficients, so a density must
be assumed. Provided that the same density is used in the code and in
the generation of the spectral file, its value is irrelevant and a 
conventional density of 1000 kgm${}^{-3}$ has been assumed. If spectral
data for the climatological aerosols are combined with mass-loadings
specified other than through the climatology, it is necessary to
consider whether this density is appropriate.

Data for water droplets were generated using a Mie scattering code. 
Whilst a single size distribution may be assumed for each species
of aerosol individually, size distributions for droplets vary widely,
depending on the location and moisture content of the atmosphere. Some
appropriate but variable measure of the size of droplet is required.
For radiative purposes, the
appropriate measure of size is the effective radius, $r_e$. $r_e$
may be parametrized or imposed (see section~\ref{sec:twostream}). The numbers in the
spectral file represent coefficients in a parametrization. They
are generated by running a Mie scattering code for a number of different
size distributions at a range of wavelengths, averaging the single
scattering properties across the spectral bands, weighting with an
appropriate function of frequency and then fitting using some appropriate
function of the effective radius. This may clearly be done in many
different ways, and to allow general freedom, the concept of a {\em
type} of droplet is introduced.  Data for type 1 were obtained by
using the size distributions specified by \citet{Rockel91} 
as the basis of the Mie calculations. Weighting
was carried out using a Planckian function at a temperature of 250 K,
and spectral averaging was carried out using the method of 
thin averaging (\citet{Edwards96rc}) and
the functional form of \citet{Slingo82} was used for fitting. 
Data for ice crystals have been generted analogously, using the 
size distributions given by \citet{Rockel91} 
as a basis and treating ice crystals
as spheres of ice. Fits were generated using a similar functional form.
{\em Note: In the generation of this file, the full range of size 
distributions given by \citet{Rockel91} was used, including data for 
small particles. In the longwave region, this encompasses a range of
sizes much below those for which geometrical optics applies, in some
instances giving an increase in the extinction with particle size.}

\end{enumerate}
