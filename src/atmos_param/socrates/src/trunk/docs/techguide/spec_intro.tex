\section{Introduction and General Remarks}

To cover the broad range of
frequencies encountered in atmospheric radiation, discretization in
frequency or wavelength must be considered carefully. The approach
adopted in all codes for use in general circulation models (GCMs) 
is to divide the
solar or infra-red spectral region into a number of bands, across
which all radiative quantities, except the absorption coefficients
of gases, may be considered uniform. More accurate computations can
be made if more bands are used; but this comes at increased computational
expense, and the balance to be struck between the two requirements
will depend on the application. For operational use in GCMs only
a small number of bands can be used. 

The Edwards-Slingo radiation scheme was developed to meet a range of 
varied requirements for radiative modelling, extending beyond the 
demands of the Unified Model itself. To meet this need for flexibility,
the discretization in frequency within this radiation code is not
fixed, but is set by an external file supplied by the user when the
code runs: this file is known as the {\em spectral file}. This 
flexibility facilitates assessment of the files used in the UM itself
against reference data in controlled experiments.

The generation of spectral files requires a detailed knowledge of
radiative transfer and judgements about efficiency for the application
in question. It is therefore not envisaged that users of the code
within the Unified Model in particular will generate their own files,
unless they have this knowledge and are developing a specific new
application. Instead, standard spectral files, appropriate for use
in the Unified Model are provided in a central directory, 
{\tt \$UMDIR/vn\$VN/ctldata/spectral}. When a new requirement arises,
such as the need to model the radiative effects of a new gas,
users should contact the radiation group to discuss the requirement.

The following general points should be noted.

\begin{enumerate}

\item
A spectral file is not an ancillary file: it contains no geographical
information, but refers to the discretization in frequency.

\item
A spectral file released with a certain version of the Unified Model
will normally be compatible with future releases (but the reverse is
not generally true because of the possible addition of data to deliver
new functionality). This makes upgrading to newer releases of the Model
simpler than it would otherwise be. A caveat to this point is that at
version 8.6 of the Unified Model the format of the spectral files was
changed from a namelist to a readable text file as used by the offline
Edwards-Slingo radiation code suite. Utilities are available within
the offline suite to convert between namelist and text versions of the
spectral files.

\item
{\bf Naming Convention}

Although there is no formal requirement to adopt a particular 
naming convention, the names of shortwave spectral files begin with the
string {\tt sp\_sw} and those of longwave spectral files with the
string {\tt sp\_lw}. (Spectral files in namelist format as used with the
UM prior to version 8.6 begin with the string {\tt spec\_} or
{\tt spec3a\_}.) When a new release of the model is prepared,
the UM team copy all existing spectral files from the old release to the
new one without change, unless advised differently by the
radiation group.

As new functionality is developed, it is sometimes necessary to change
spectral files, either by adding new material or by replacing old
material. If new material is added, in such a way that the results of
existing runs are unchanged at the new release, the name of the file
is not changed; but if existing data are altered, a new name is used.
An example may make this clearer. Suppose that at version 5.11 of the
UM we have a file {\tt spec3a\_sw\_orig} and that for version 5.12 a new
scheme is to be added to the UM which requires the radiative modelling
of volcanic ash. This would require the addition of information to
the spectral file, but the new material would not change the results
of any run which could be carried out at both versions 5.11 and 5.12:
to allow for simple upgrading, the new material would be added to the
spectral files for 5.12 without any change of name. Suppose now that
for version 5.13 improvements to the modelling of ash had been made
and that the data in the spectral file needed to be changed. This
would represent a modification to an existing capability, so the old
spectral file would be copied to the new directory for 5.13 without
new data to allow existing configurations to continue, 
but a new spectral file with the revised data and a new name
would be introduced. If the old file were revised without a change of
name it would be possible to upgrade an existing experiment and get
different results: that would be unacceptable.

\item
{\bf Generation of and Additions to Spectral Files}

As noted before, the generation or alteration of spectral files requires
expertise in radiative transfer. The responsibility for changes to the
standard files lies with the radiation group. Most users will
never need to alter a spectral file, but it is possible that some
users of the portable model with expertise in radiative transfer may wish 
to generate files for their own specific purposes and the next paragraph 
is addressed to them.

Spectral files are generated using the pre-processing suite in the 
off-line version of the radiation code, which is maintained by the
radiation team. It is possible to generate a spectral file form scratch,
but more usually, the requirement is to add to or modify an existing file.
Prior to version 8.6 spectral files were read into
the UM as namelists. The current UM along with the off-line code uses
a more readable text format. Conversion from a namelist to the text
format is carried out with the program {\tt nml\_spec} of the external
suite and in the reverse direction with the program {\tt spec\_nml}. 

Data on aerosols are generated in consultation with the aerosol 
modelling group.

\end{enumerate}




